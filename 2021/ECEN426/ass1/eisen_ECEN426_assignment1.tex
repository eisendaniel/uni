%----------------------------------------------------------------------------------------
%	PACKAGES AND DOCUMENT CONFIGURATIONS
%----------------------------------------------------------------------------------------
\documentclass[11pt]{article}
\usepackage{amsmath} % Required for some math elements
\usepackage{hyperref}
\usepackage[table,xcdraw]{xcolor}
\usepackage{lipsum} 
\usepackage{cite}
\usepackage{graphicx} % Required for the inclusion of images
\usepackage{algorithmic}
\usepackage{array}
\usepackage{adjustbox}
\usepackage{bookmark}
\usepackage[margin=24mm]{geometry}


\interdisplaylinepenalty=2500 %Note that the amsmath package sets \interdisplaylinepenalty to 10000 thus preventing page breaks from occurring within multiline equations. Use: \interdisplaylinepenalty=2500 after loading amsmath to restore such page breaks as IEEEtran.cls normally does

\hypersetup{ %color attributes of citation, link, etc.
    colorlinks=orange,
    linkcolor=cyan,
    filecolor=gray,      
    urlcolor=cyan,
    citecolor=cyan,
}
%----------------------------------------------------------------------------------------
%	DOCUMENT INFORMATION
%----------------------------------------------------------------------------------------
\title{ECEN426 Assignment 1 \\ Mechatronic Picker \& Dampers Literature Review}
\author{Daniel Eisen}
\date{\today}

\begin{document}
\maketitle
%----------------------------------------------------------------------------------------
%	DOCUMENT CONTENT
%----------------------------------------------------------------------------------------
\section{Overview}
Core to the design of mechatronic chordophones is the string excitation and damping "loop". This pattern is necessary to the (re)production of music by enabling the how long a note is sustained by the excited string.

This report explores a collection of projects that have developed picker and damper mechanisms, compare and comment on their approaches and attempt to better understand breadth of the field.

While the traditional approach to string excitation is the emulation of "human" style picking/strumming is the most common, this niche area does have some projects that are utilising the unique aspects of medium to produce a wider field of sound via interesting takes on these core mechanisms.


\section{Mechanisms}



\subsection{Picking/Excitation}
The picking or excitation mechanism is the means through which energy is primarily imparted into the string to produce sound. This mechanism, depending on its design, can control/change aspects of the produced sound, i.e. rhythm, speed, timbre, pitch.

A straight-forward approach is linear picking. This describes the method where a guitar pick is attached to an actuator that moves through a straight path across the string(s). These solutions are usually easy to mount and drive, and as shown with the case of the Actuated Guitar\cite{actuated}, can extend to multi-string strumming. This method of course lacks any sound control, as the motion is so heavily simplified when compared the a hands range.


A rotary picking motion, especially when driven by stepper or servo motors, is a fast and accurate (as well as generally popular way) of picking a string. The LEMUR GuitarBot\cite{singer} uses an example is a 'pickwheel', where multiple picks are mounted servo driven shaft. This allows a pick accurately rotated so that a note can be struck and the next
pick brought into position to await the next trigger. Additionally pickwheels can be rotated at various speeds to vary pitch and produce unique tremolo.
It however lacks any pick contact area control to affect excitation amplitude. The StrumBot\cite{strumbot}, provides a method for this with a SCARA arm picking mechanism with servo control pitch angle (of 2 mounted picks) for amplitude control. Able to play different distances from bridge back and forth at 9.3 strums/s. 


Pick emulation is however not the only path explored this field. Where previous methods attempt the remain close the sound of pick on string with various degrees of freedom, other projects make uses the mechatronic field to take novel approaches.
The RAEG\cite{hammer} attaches hammering mechanisms to the body of a guitar (near the bridge) and is design such that a human player is still able to have shared control (i.e. strum while bot hammers). Servos drive a striking hammer per string which enables unique excitation, as well as amplitude control per strike.
The Tremolo-Harp\cite{vibe_harp} takes an even more "out-there" approach. The strings are actuated directly by suspended vibration motors, these allow for sharp pulses to long drawn out sustains in order the create very unique waveforms/timbre. 
While not directly comparable to the pick emulation techniques, these illustrate the breadth of possibly within the mechatronic chordophone space.

\subsection{Damping/Sustain Control}
Damping mechanisms are often simpler than their excitation counterparts (where much of the sound exploration is done) so the variation is lesser. 

A very common method is just to have binary sustain control via a solenoid, such as shown with the LEMUR\cite{singer}. But the Tremolo-Harp\cite{vibe_harp} shows these can be utilised as a method of "hammer" like excitation as well as sustain control. 

The RAEG\cite{hammer} matches its hammer control with more high speed servos that drive a damper with rubber interface per string. There control the damping pressure the control drop-off and various interfaces can control the "sound" of that damping curve.  

The StrumBot\cite{strumbot} displays the most unique damping mechanism that doubles as variable tension control with built in muting contact to control sustain. 


\section{Discussion}
From the project explored above, it can be said there is a large variation in picking/excitation approaches, that also extend beyond traditional "picking" excitation of string approaches. Most mechanisms that emulate piking often forgo pick area contact control and opt for variable strike speed (pitch over volume), but when they do the mechatronic is more greater in complexity but can provide greater functionality that just amplitude control\cite{strumbot}. Damping mechanisms are often simpler or given less attention with common  approach of simple digital solenoids, but there are options that allow for damping pressure control, and can even be built into compound pitch control mechanisms.

Definitive area of lacking more comparable results between mechatronic implementations, but the field is rich and experimental.


\newpage
\bibliography{ref}
\bibliographystyle{IEEEtran}
\end{document}
