%----------------------------------------------------------------------------------------
%	PACKAGES AND DOCUMENT CONFIGURATIONS
%----------------------------------------------------------------------------------------
\documentclass[11pt]{article}
\usepackage{amsmath} % Required for some math elements
\usepackage{hyperref} 
\usepackage{xcolor}
\usepackage{lipsum} 
\usepackage{cite}
\usepackage{graphicx} % Required for the inclusion of images
\usepackage{algorithmic}
\usepackage{array}
\usepackage{bookmark}
\usepackage{listings}
\usepackage{amssymb}
\usepackage{enumitem}
\usepackage[margin=16mm]{geometry}
\usepackage[caption=false, font=footnotesize]{subfig}
\usepackage{fancyhdr}
\renewcommand{\headrulewidth}{0.4pt}
\renewcommand{\footrulewidth}{0.4pt}

\usepackage[active,tightpage]{preview}
\renewcommand{\PreviewBorder}{1in}
\newcommand{\Newpage}{\end{preview}\begin{preview}}

\newlist{steps}{enumerate}{1}
\setlist[steps, 1]{label = Step \arabic*:}

\hypersetup{ %color attributes of citation, link, etc.
    colorlinks=true,
    linkcolor=blue,
    filecolor=gray,      
    urlcolor=blue,
    citecolor=blue,
}

\newcommand{\matlab}{\textsc{Matlab }} %very important and totally necessary addition

\newcommand\Item[1][]{%
  \ifx\relax#1\relax  \item \else \item[#1] \fi
  \abovedisplayskip=0pt\abovedisplayshortskip=0pt~\vspace*{-\baselineskip}}
%----------------------------------------------------------------------------------------
%	DOCUMENT INFORMATION
%----------------------------------------------------------------------------------------

\title{ECEN 415 \\ Assignment 1 Submission}
\author{Daniel Eisen : 300447549}
\date{\today}

\begin{document}
\begin{preview}
\maketitle

%----------------------------------------------------------------------------------------
%	DOCUMENT CONTENT
%----------------------------------------------------------------------------------------
\section*{Section A - Formative Questions}
\begin{enumerate}
    \item 
    \begin{enumerate}
        \item 
        $$G_1(s) = \frac{20(s^2 + s + 0.5)}{s(s+1)(s+10)}$$
        \begin{center}
            \includegraphics[width=0.33\textwidth]{fig/1a.png}
        \end{center}
        The system in its current state is stable as there are no enclosures of the critical point, and no open open-loop poles in the right half side of the s-place. Neither increasing decreasing the gain of this system will result in an enclosure, and thus cannot be made unstable with this method.  
        \item 
        $$G_2(s) = \frac{20(s^2 + s + 0.5)}{s(s-1)(s+10)}$$
        \begin{center}
            \includegraphics[width=0.33\textwidth]{fig/1b.png}
        \end{center}
        The system in its current state is stable as there is one open-loop pole in the right half side of the s-place and one anti-clockwise encirclement of the critical point. However with reduced gain, there will be no enclosure of the critical point and the system can be driven unstable.
        \item 
        $$G_3(s) = \frac{s^2 + 3}{(s+1)^2}$$
        \begin{center}
            \includegraphics[width=0.33\textwidth]{fig/1c.png}
        \end{center}
        The system in its current state is stable as there are no enclosures of the critical point, and no open open-loop poles in the right half side of the s-place. Neither increasing decreasing the gain of this system will result in an enclosure, and thus cannot be made unstable with this method.
        \item 
        $$G_4(s) = \frac{3(s+1)}{s(s-10)}$$
        \begin{center}
            \includegraphics[width=0.33\textwidth]{fig/1d.png}
        \end{center}
        The system is currently unstable, as there is one open-loop pole in the right half side of the s-place and no anti-clockwise encirclements of the critical point. By increasing the gain we can make and anti-clockwise encirclement of the critical point and result in a stable system.
    \end{enumerate}
    \item
    $$G = e^{-0.2s}\frac{4}{s+2}$$
    \begin{enumerate}
        \item 
        \begin{center}
            \includegraphics[width=0.33\textwidth]{fig/2a.png}
        \end{center}
        \item 
        \begin{center}
            \includegraphics[width=0.33\textwidth]{fig/2bs.png}
            \includegraphics[width=0.33\textwidth]{fig/2bu.png}
        \end{center}
        \item 
        \begin{center}
            \includegraphics[width=0.33\textwidth]{fig/2c.png}
        \end{center}
    \end{enumerate} 
    \item 
    \begin{center}
        \includegraphics[height=0.3\textwidth]{fig/3_imp.png}
        \includegraphics[height=0.3\textwidth]{fig/3_stp.png}
        \includegraphics[height=0.3\textwidth]{fig/3_bode.png}
    \end{center}
\end{enumerate}

\section*{Section B - Summative Questions}
\begin{enumerate}
    \item 
    \begin{enumerate}
        \item 
        \item 
        \item 
    \end{enumerate}
    \item 
    \item 
\end{enumerate}
\end{preview}
\end{document}