%----------------------------------------------------------------------------------------
%	PACKAGES AND DOCUMENT CONFIGURATIONS
%----------------------------------------------------------------------------------------
\documentclass[11pt]{article}
\usepackage{amsmath} % Required for some math elements
\usepackage{hyperref} 
\usepackage{xcolor}
\usepackage{lipsum} 
\usepackage{cite}
\usepackage{graphicx} % Required for the inclusion of images
\usepackage{algorithmic}
\usepackage{array}
\usepackage{bookmark}
\usepackage{listings}
\usepackage{amssymb}
\usepackage{enumitem}
\usepackage[margin=8mm]{geometry}
\usepackage[caption=false, font=footnotesize]{subfig}
\usepackage{fancyhdr}
\pagestyle{fancy}
\lhead{ENGR222 Assignment 1 Submission}
\rhead{Daniel Eisen : 300447549}
\cfoot{\thepage}
\renewcommand{\headrulewidth}{0.4pt}
\renewcommand{\footrulewidth}{0.4pt}
\usepackage[active,tightpage]{preview}

\renewcommand{\PreviewBorder}{1in}

\newlist{steps}{enumerate}{1}
\setlist[steps, 1]{label = Step \arabic*:}

\hypersetup{ %color attributes of citation, link, etc.
    colorlinks=true,
    linkcolor=blue,
    filecolor=gray,      
    urlcolor=blue,
    citecolor=blue,
}

\newcommand{\matlab}{\textsc{Matlab }} %very important and totally necessary addition
 
\newcommand\Item[1][]{%
  \ifx\relax#1\relax  \item \else \item[#1] \fi
  \abovedisplayskip=0pt\abovedisplayshortskip=0pt~\vspace*{-\baselineskip}}
%----------------------------------------------------------------------------------------
%	DOCUMENT INFORMATION
%----------------------------------------------------------------------------------------

\title{ENGR 222 \\ Assignment 2 Submission}
\author{Daniel Eisen : 300447549}
\date{\today}

\begin{document}
\begin{preview}

\maketitle
%----------------------------------------------------------------------------------------
%	DOCUMENT CONTENT
%----------------------------------------------------------------------------------------

\begin{enumerate}
    \item \textbf{Multivariate Function}
          $$f(x,y) = -2x^{3} + 3x^{2}y + 2y^{3} - 9y + 5$$
          \begin{enumerate}
              \Item
              \begin{align*}
                  f_{x} & = -6x^{2} + 6xy       \\
                  f_{y} & = 3x^{2} + 6y^{2} - 9 \\
              \end{align*}
              \Item
              \begin{align*}
                  f_{xy} & = 6x        \\
                  f_{xx} & = -12x + 6y \\
                  f_{yy} & = 12y       \\
              \end{align*}
              \Item
              \begin{align*}
                  f_x                              & = -6x^{2} + 6xy = 0                                  \\
                  f_y                              & = 3x^{2} + 6y^{2} - 9 = 0                            \\
                  \mathrm{by \; inspection \;}  (x & =y=1,-1)                                             \\
                  for \; x=0,                                                                             \\
                  f_x                              & = 0                                                  \\
                  f_y                              & = 6y^2 - 9 = 0                                       \\
                  \therefore \; y = \sqrt{9/6}     & = \sqrt{\frac{3}{2}}                                 \\
                  for \; y=0:                                                                             \\
                  f_x                              & = -6x^2 = 0                                          \\
                  f_y                              & = 3x^2 - 9 = 0                                       \\
                  \mathrm{no \; x}                                                                        \\
                  \mathrm{critical \; points}      & \Rightarrow [(1,1), (-1,-1), (0,\sqrt{\frac{3}{2}})] \\
              \end{align*}
              \Item Second Partials test:
              \begin{align*}
                  D                                        & = f_{xx}(0,\sqrt{\frac{3}{2}}) \cdot f_{yy}(0,\sqrt{\frac{3}{2}}) - f^2_{xy}(0,\sqrt{\frac{3}{2}}) \\
                  f_{xx} = -12x + 6y, f_{yy} = 12y, f_{xy} & = 6x                                                                                               \\
                  D                                        & = (-12(0) + 6\left(\sqrt{\frac{3}{2}}\right))(12\left(\sqrt{\frac{3}{2}}\right)) - (6(0))^2        \\
                                                           & = (0 + 3\sqrt{6})(6\sqrt{6})-0                                                                     \\
                                                           & =    108
              \end{align*}
              $D>0$ and $f_{xx} > 0$ therefore, this critical point is a local minimum.
          \end{enumerate}

          \newpage
    \item \textbf{Quick questions}
          \begin{enumerate}
              \Item
              $f(x,y,z) = e^{x}cos(y)(1-z)^{2}, \; \textbf{u} = (0.36, 0.48, 0.8)$ \\
              \begin{align*}
                  D_{\textbf{\textbf{u}}} & = f_{x}u_{1} + f_{y}u_{2} + f_{z}u_{3} \\ \\
                  f_{x}                   & = e^{x}cos(y)(1-z)^{2}                 \\
                  f_{x}(0,0,0)            & = 1 \times 1 \times 1 = 1              \\
                  f_{y}                   & = -e^{x}sin(y)(1-z)^2                  \\
                  f_{y}(0,0,0)            & = -1 \times 0 \times 1 = 0             \\
                  f_{z}                   & = 2e^{x}cos(y)(z - 1)                  \\
                  f_{z}(0,0,0)            & = 2 \times 1 \times -1 = -2            \\ \\
                  D_{\textbf{\textbf{u}}} & = 1(0.36) + 0(0.48) + -2(0.8)=-1.24    \\
              \end{align*}
              \Item
              $f(x,y,z) = (1+x)(1-y^2)(1-z)^2, \; \textbf{p} = (1,2,3) $\\
              \begin{align*}
                  L(x,y,z)          & = f(x_0, y_0, z_0) + f_x(x_0, y_0, z_0)(x-x_0) \\
                                    & + f_y(x_0, y_0, z_0)(y-y_0)                    \\
                                    & + f_z(x_0, y_0, z_0)(z-z_0)                    \\\\
                  f(\textbf{p})     & = (1+1)(1-2^2)(1-3)^2=-24                      \\ \\
                  f_{x}             & = (1-y^2)(1-z)^2                               \\
                  f_{x}(\textbf{p}) & = (1-2^2)(1-3)^2=-12                           \\
                  f_{y}             & = (1+x)(-2y)(1-z)^2                            \\
                  f_{y}(\textbf{p}) & = (1+1)(-2(2))(1-3)^2=-32                      \\
                  f_{z}             & = 2(1+x)(1-y^2)(z-1)                           \\
                  f_{z}(\textbf{p}) & = 2(1+1)(1-2^2)(3-1)=-24                       \\\\
                  L(\textbf{p})     & = -24 + (-12)(x-1) + (-32)(y-2) + (-24)(z-3) \\
                  &= 124 - 12x - 32y - 24z
              \end{align*}
              \Item $f(x,y) = e^{-x^2-y^2} = e^{-x^2}e^{-y^2}, \; \textbf{p} = (1,1) $\\
              \begin{align*}
                L(x,y) &=  f(\textbf{p}) + f_x(\textbf{p})(x-x_0)+ f_y(\textbf{p})(y-y_0) \\
                p_{2}(x,y) &= L(x,y) + \frac{1}{2}\left[(x-x_0)^{2}f_{xx}(\textbf{p}) + 2(x-x_0)(y-y_0)f_{xy}(\textbf{p}) + (y-y_0)^{2}f_{yy}(\textbf{p})\right]\\
              \end{align*}
              \begin{align*}
                  f_{x}  &= -2xe^{-x^2}e^{-y^2} \\
                  &= -2xe^{-x^2-y^2} \\
                  f_{y}  &= -2ye^{-x^2}e^{-y^2} \\
                  &= -2ye^{-x^2-y^2}\\\\
                  f_{xx} &= e^{-y^2}(-2(e^{-x^2}) + -2x(-2xe^{-x^2}))\\
                  & = (4x^2 - 2)e^{-x^2-y^2} \\
                  f_{yy} & = (4y^2 - 2)e^{-x^2-y^2} \\
                  f_{xy} &= -2xye^{-x^2-y^2}\\\\ 
                  L(\textbf{p}) &= \\
                  p_{2}(\textbf{p}) &= \\
              \end{align*}
              \Item \\
              \Item \\
          \end{enumerate}
    \item \textbf{Double integrals}
          \begin{enumerate}
              \item
              \item
              \item
              \item
              \item
          \end{enumerate}
    \item \textbf{Lab question}
          \begin{enumerate}
              \item
                    \begin{enumerate}
                        \item
                        \item
                        \item
                    \end{enumerate}
              \item
                    \begin{enumerate}
                        \item
                        \item
                        \item
                    \end{enumerate}
          \end{enumerate}
\end{enumerate}
\end{preview}

\end{document}