%----------------------------------------------------------------------------------------
%	PACKAGES AND DOCUMENT CONFIGURATIONS
%----------------------------------------------------------------------------------------
\documentclass[11pt]{article}
\usepackage{amsmath} % Required for some math elements
\usepackage{hyperref} 
\usepackage{xcolor}
\usepackage{lipsum} 
\usepackage{cite}
\usepackage{graphicx} % Required for the inclusion of images
\usepackage{algorithmic}
\usepackage{array}
\usepackage{bookmark}
\usepackage{listings}
\usepackage{amssymb}
\usepackage{enumitem}
\usepackage[margin=16mm]{geometry}
\usepackage[caption=false, font=footnotesize]{subfig}
\usepackage{fancyhdr}
\pagestyle{fancy}
\lhead{ENGR222 Assignment x Submission}
\rhead{Daniel Eisen : 300447549}
\cfoot{\thepage}
\renewcommand{\headrulewidth}{0.4pt}
\renewcommand{\footrulewidth}{0.4pt}

\newlist{steps}{enumerate}{1}
\setlist[steps, 1]{label = Step \arabic*:}

\hypersetup{ %color attributes of citation, link, etc.
    colorlinks=true,
    linkcolor=blue,
    filecolor=gray,      
    urlcolor=blue,
    citecolor=blue,
}

\newcommand{\matlab}{\textsc{Matlab }} %very important and totally necessary addition

\newcommand\Item[1][]{%
  \ifx\relax#1\relax  \item \else \item[#1] \fi
  \abovedisplayskip=0pt\abovedisplayshortskip=0pt~\vspace*{-\baselineskip}}
%----------------------------------------------------------------------------------------
%	DOCUMENT INFORMATION
%----------------------------------------------------------------------------------------

\title{ENGR 222 \\ Assignment x Submission}
\author{Daniel Eisen : 300447549}
\date{\today}

\begin{document}
\maketitle
%----------------------------------------------------------------------------------------
%	DOCUMENT CONTENT
%----------------------------------------------------------------------------------------
\begin{enumerate}
    \item 
    \begin{enumerate}
        \Item\\
        \Item\\
        \Item\\
        \Item\\
        \Item\\
    \end{enumerate}
    \item 
    \begin{enumerate}
        \Item\\
        \Item\\
        \Item\\
        \Item\\
        \Item\\
    \end{enumerate}
    \item 
    \begin{enumerate}
        \Item\\
        \Item\\
        \Item\\
        \Item\\
        \Item\\
    \end{enumerate}
    \item 
    \begin{enumerate}
        \Item\\
        \Item\\
        \Item\\
        \Item\\
        \Item\\
    \end{enumerate}
    \item 
    \begin{enumerate}
        \Item\\
        \Item\\
        \Item\\
        \Item\\
        \Item\\
    \end{enumerate}
\end{enumerate}
\end{document}