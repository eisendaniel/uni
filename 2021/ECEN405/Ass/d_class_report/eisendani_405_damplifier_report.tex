 %----------------------------------------------------------------------------------------
%	PACKAGES AND DOCUMENT CONFIGURATIONS
%----------------------------------------------------------------------------------------
\documentclass[11pt]{article}
\usepackage{subcaption}
\usepackage{amsmath} % Required for some math elements
\usepackage{hyperref} 
\usepackage[table,xcdraw]{xcolor}
\usepackage{lipsum} 
\usepackage{cite}
\usepackage{graphicx} % Required for the inclusion of images
\usepackage{algorithmic}
\usepackage{array}
\usepackage{bookmark}
\usepackage{listings}
\usepackage{amssymb}
\usepackage{enumitem}
\usepackage{pythonhighlight}
\usepackage[T1]{fontenc}
\usepackage{inconsolata}
\usepackage[margin=16mm]{geometry}
\usepackage{cleveref}

\newlist{steps}{enumerate}{1}
\setlist[steps, 1]{label = Step \arabic*:}

\hypersetup{ %color attributes of citation, link, etc.
    colorlinks=true,
    linkcolor=blue,
    filecolor=gray,      
    urlcolor=blue,
    citecolor=blue,
}

\newcommand{\matlab}{\textsc{Matlab }} %very important and totally necessary addition
\newcommand{\hdotrule}[1]{\hbox to \textwidth{\leaders\hbox to #1pt{\hss . \hss}\hfil}}

\newcommand\Item[1][]{%
  \ifx\relax#1\relax  \item \else \item[#1] \fi
  \abovedisplayskip=0pt\abovedisplayshortskip=0pt~\vspace*{-\baselineskip}}
%----------------------------------------------------------------------------------------
%	DOCUMENT INFORMATION
%----------------------------------------------------------------------------------------

%Here is a little bit of guidance for the report, it should follow this basic structure, or at least cover the topics mentioned here (if you have a different structure in mind). I would expect the report to be < 10 pages, but as there has been no page limit discussed before now, there will be no marks associated with length.

\title{ECEN 405 -  D Class Amplifier \\ \large{\textit{``What a buck convertor would say if it could talk''}} }
\author{Daniel Eisen : 300447549 \\Team members: Niels CLayton \& Nickolai Wolfe}
\date{}

\begin{document}
\maketitle
\begin{center}
  \vspace*{-10mm}
  \includegraphics[width=0.75\textwidth]{img/real.png}
\end{center}{

\section{Introduction}

In the relm of auido ampliifers the D Class offers a method of supplying a high power loads with very high efficiency (esspecailly when compared to other classes). Classes A, AB offer high power output with very low signal distortion but suffer greatly in the efficency department. The D class of ampliifers can achieved up to 90-95\% power efficiency by taking advantage of 

\subsection*{Specifications}\label{S:spec}
\begin{itemize}
  \item $P_{out}=80W$ for $R_{L} = 4\Omega$
  \item 10Hz to 200Hz Bandwidth
  \item Input sensitivity of 1V for maximum output (interperated as 1V amplitude, 2V pk-pk)
  \item Maximum costs: \$50 per person
\end{itemize}

\section{Design}

\textit{Here you should describe how your class D amplifier works, giving details of each subsection.
In detail, you should describe the section you designed and the design choices you made. If your team broke up the design of the amplifier in a way that doesn’t suit individual parts being discussed, you will need to talk about the whole design in a bit more detail but you should also describe how the work was delegated and why.}
 
\begin{figure}[h!]
  \centering
  \frame{\includegraphics[page=1, trim={18mm 132mm 2mm 10mm},clip,width=0.85\textwidth]{img/schematic.pdf}}  \caption{Top Level Design Schematic}
\end{figure}

\subsection{Input Bandpass Filter}

Active Bandpass filter with adjustable gain stage to pass 10Hz to 200Hz.

\begin{figure}[h!]
  \centering
  \frame{\includegraphics[page=2, trim={25mm 70mm 90mm 60mm},clip,width=0.85\textwidth]{img/schematic.pdf}}
  \caption{Input filtering schematic}
\end{figure}

\subsection{Audio Sampling and SPWM}

\begin{figure}[h!]
  \centering
  \frame{\includegraphics[page=3, trim={35mm 133.5mm 30mm 15mm},clip,width=0.85\textwidth]{img/schematic.pdf}}
  \caption{Sampling triangle wave \& SPWM generation schematic}
\end{figure}

\subsection{Power Stage and Output Filter}

\begin{figure}[h!]
  \centering
  \frame{\includegraphics[page=5, trim={85mm 149mm 5mm 12mm},clip,width=0.85\textwidth]{img/schematic.pdf}}
  \caption{Gate driver schematic}
\end{figure}


\begin{figure}[h!]
  \centering
  \includegraphics[width=0.75\textwidth]{img/output_filter_sim.png}
  \caption{Output filter option simulations}
\end{figure}

\begin{figure}[h!]
  \centering
  \frame{\includegraphics[page=4, trim={115mm 97mm 105mm 75mm},clip,width=0.6\textwidth]{img/schematic.pdf}}
  \caption{Final output filter schematic}
\end{figure}


\section{Implementation}
 
\textit{Here you should discuss the assembly of the amplifier and any problems you faced as a team building the amplifier.
Here, the individual components should also be characterised. For example: if you have a filter, what is the response and how does it compare to the calculated? If you have a triangle wave, how does it look? Is it doing what I should? Why? Why not? How do the inputs/outputs of your comparator look? How does the square wave on the gate of the MOSFETs look?}
 

\subsection{PCB Layout}

\subsection{Gate Driving and Bridge Output}
\begin{figure}[h!]
  \centering
  \begin{subfigure}{0.3\textwidth}
      \includegraphics[width=\columnwidth]{img/testing/power_output/gate_input.JPG}
      \subcaption{High and Low side gate signals}
  \end{subfigure}
  \begin{subfigure}{0.3\textwidth}
      \includegraphics[width=\columnwidth]{img/testing/power_output/dead_time.JPG}
      \subcaption{Deadtime}
  \end{subfigure}
  \begin{subfigure}{0.3\textwidth}
      \includegraphics[width=\columnwidth]{img/testing/power_output/fet_output.JPG}
      \subcaption{Both bridges amplified SPWM (FET outputs)}
  \end{subfigure}
  \caption{}
\end{figure}


\section{Results}
 
\textit{Here I would expect to see the results of the whole amp, for example: an output wave, analysis of the efficiency, discuss maximum power output (which may be frequency dependent), and THD.} 
 
\begin{figure}[h!]
  \centering
  \begin{subfigure}{0.3\textwidth}
      \includegraphics[width=\columnwidth]{img/testing/power_output/filter_output_1Hz.JPG}
      \subcaption{}
  \end{subfigure}
  \begin{subfigure}{0.3\textwidth}
      \includegraphics[width=\columnwidth]{img/testing/power_output/filter_output_10Hz.JPG}
      \subcaption{}
  \end{subfigure}
  \begin{subfigure}{0.3\textwidth}
      \includegraphics[width=\columnwidth]{img/testing/power_output/filter_output_100Hz.JPG}
      \subcaption{}
  \end{subfigure}
  \begin{subfigure}{0.3\textwidth}
      \includegraphics[width=\columnwidth]{img/testing/power_output/filter_output_200Hz.JPG}
      \subcaption{}
  \end{subfigure}
  \begin{subfigure}{0.3\textwidth}
      \includegraphics[width=\columnwidth]{img/testing/power_output/filter_output_500Hz.JPG}
      \subcaption{}
  \end{subfigure}
  \begin{subfigure}{0.3\textwidth}
      \includegraphics[width=\columnwidth]{img/testing/power_output/filter_output_2kHz.JPG}
      \subcaption{}
  \end{subfigure}
  \caption{}
\end{figure}

\begin{figure}[h!]
  \centering
  \includegraphics[width=0.75\textwidth]{img/testing/amplifier_bode.pdf}
  \caption{}
\end{figure}

\begin{figure}[h!]
  \centering
  \includegraphics[width=0.75\textwidth]{img/testing/Efficiency_vs._Frequency.pdf}
  \caption{}
\end{figure}

\begin{table}[h!]
  \centering
  \begin{tabular}{l|l}
  \rowcolor[HTML]{E0E0E0} 
  \textbf{Frequency (Hz)} & \textbf{THD (\%)} \\ \hline
  30                 & 1.8               \\
  50                 & 2.2               \\
  100                & 3.2               \\
  200                & 3.3               \\
  300                & 3.5               \\
  500                & 3.2              
  \end{tabular}
  \caption{Output total harmonic distortion across frequency}
  \label{T:THD}
\end{table}


\section{Conclusions}
 
\textit{What worked, didn’t work? How would you change your approach? Any interesting insights?}

\newpage
\section*{Appendix}

\subsection*{Input Filter}
\begin{figure}[h!]
  \centering
  \begin{subfigure}{0.3\textwidth}
    \includegraphics[width=\columnwidth]{img/testing/input_filter/input_1Hz.JPG}
    \subcaption{1Hz}
  \end{subfigure}
  \begin{subfigure}{0.3\textwidth}
    \includegraphics[width=\columnwidth]{img/testing/input_filter/input_10Hz.JPG}
    \subcaption{10Hz}
  \end{subfigure}
  \begin{subfigure}{0.3\textwidth}
    \includegraphics[width=\columnwidth]{img/testing/input_filter/input_100Hz.JPG}
    \subcaption{100Hz}
  \end{subfigure}
  \begin{subfigure}{0.3\textwidth}
    \includegraphics[width=\columnwidth]{img/testing/input_filter/input_200Hz.JPG}
    \subcaption{200Hz}
  \end{subfigure}
  \begin{subfigure}{0.3\textwidth}
    \includegraphics[width=\columnwidth]{img/testing/input_filter/input_500Hz.JPG}
    \subcaption{500Hz}
  \end{subfigure}
  \begin{subfigure}{0.3\textwidth}
    \includegraphics[width=\columnwidth]{img/testing/input_filter/input_2kHz.JPG}
    \subcaption{2kHz}
  \end{subfigure}
\end{figure}

\subsection*{Sampling/SPWM}
\begin{figure}[h!]
  \centering
  \begin{subfigure}{0.3\textwidth}
    \includegraphics[width=\columnwidth]{img/testing/spwm/triangle_wave_32kHz.JPG}
    \subcaption{Triangle Qave}
  \end{subfigure}
  \begin{subfigure}{0.3\textwidth}
    \includegraphics[width=\columnwidth]{img/testing/spwm/spwm_no_input.JPG}
    \subcaption{SPWM w/ 0V Input}
  \end{subfigure}\\
  \begin{subfigure}{0.3\textwidth}
    \includegraphics[width=\columnwidth]{img/testing/spwm/input_sampling_0.JPG}
    \subcaption{Input Signal Sampling Far}
  \end{subfigure}
  \begin{subfigure}{0.3\textwidth}
    \includegraphics[width=\columnwidth]{img/testing/spwm/input_sampling_1.JPG}
    \subcaption{Input Signal Sampling Near}
  \end{subfigure}
\end{figure}

\end{document}