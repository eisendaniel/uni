\documentclass[a4paper,11pt]{article}
\usepackage[left=2.5cm, right=2.5cm, top=1.5cm, bottom=1.5cm]{geometry}
\usepackage{graphicx}
\usepackage{amssymb}
\usepackage{amsmath}
\usepackage[procnames]{listings}
\usepackage{xcolor}
\usepackage[active,tightpage]{preview}
\usepackage{hyperref}

\hypersetup{ %color attributes of citation, link, etc.
    colorlinks=true,
    linkcolor=blue,
    filecolor=gray,
    urlcolor=blue,
    citecolor=blue,
}

\setlength{\parindent}{0pt}

\renewcommand{\PreviewBorder}{1in}
\newcommand{\Newpage}{\end{preview}\begin{preview}}
\newcommand{\matlab}{\textsc{Matlab}} %very important and totally necessary addition
\newcommand{\parallelsum}{\mathbin{\!/\mkern-5mu/\!}}

%'codify' text for snippets
\usepackage{xcolor}
\definecolor{codegray}{gray}{1}
\newcommand{\code}[1]{\colorbox{codegray}{\texttt{#1}}}

\definecolor{keywords}{RGB}{255,0,90}
\definecolor{comments}{RGB}{0,0,113}
\definecolor{p_red}{RGB}{160,0,0}
\definecolor{p_green}{RGB}{0,150,0} 
\lstset{language=Python, 
        basicstyle=\ttfamily\small, 
        keywordstyle=\color{keywords},
        commentstyle=\color{comments},
        stringstyle=\color{p_red},
        showstringspaces=false,
        identifierstyle=\color{p_green},
		procnamekeys={def,class}}

\graphicspath{ {./images/} }
           
\begin{document}
\begin{preview}
\title{\LARGE{\textbf{ECEN 321 Lab 5\\}}Regression}
\author{Niels Clayton : 300437590}
\date{}
\maketitle
\hrule

\section*{Introduction}

Regression is a commonly used tool for assessing the relationship between variables. It is often used to compute the line of best fit of a dataset, and can even be used to help predict future values of a dataset. In this Lab we will be performing a regression on temperature anomaly data from a collection of locations around the globe. We will attempt to use this regression to come to a conclusion on if the average global temperature is increasing, decreasing, or remaining static.

\section*{Theory}

For a simple linear regression, the first step is to calculate the correlations coefficient of the data sets, often denoted $r$. This is done with the following equation:

% Equation align
\begin{align*}
    r &= \frac{1}{n-1}\sum_{i = 1}^{n} \left(\frac{x_i-\bar{x}}{s_x} \right) \left(\frac{y_i-\bar{y}}{s_y}\right)\\\\
      &= \frac{\sum_{i = 1}^{n}  (x_i - \bar{x})(y_i - \bar{y})  }{\sqrt{\sum_{i = 1}^{n}  (x_i - \bar{x})^2(y_i - \bar{y})^2 }}\\
\end{align*}

Using the above equation we are able to calculate the correlation coefficient between variable $x$ and variable $y$. From the correlation coefficient, we are then able to calculate the slope of the line of best fit, and pot it against the data using the following equation:

$$ \beta_1 = r \frac{s_y}{s_x} $$

Once this correlation coefficient has been computed, we can compute the p-statistic of the correlation coefficient using the following equation:\\

$$U = \frac{r\sqrt{n-2}}{\sqrt{1-r^2}} \thicksim t(n-2)$$

\section*{Results}

This regression was performed on data from Wuhan China, and Wellington New Zealand, the results can be seen below in figures 1, and 2

\begin{center}
    \fbox{\includegraphics[width=0.9\textwidth]{wuhan.png}}
\end{center}

\begin{center}
    \fbox{\includegraphics[width=0.9\textwidth]{wellington.png}}
\end{center}
 
Using the null hypotheses that the correlation coefficient is zero the following was calculated:

$H_o : r = 0$\\
$H_1 : r > 0$\\

using the equation above, the p-static was calculated to be:\\

Wuhan China - $p = 0$

Wellington New Zealand - $p = 5.9\times 10^{-15} $\\

Due to the the insignificance of the p statistic for both Wuhan China, and Wellington New Zealand, we can confidently disregard the null hypotheses that $H_o : r = 0$. This would suggest that there is an increase in the average temperature in these locations.


\end{preview}
\end{document}