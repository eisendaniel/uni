%----------------------------------------------------------------------------------------
%	PACKAGES AND DOCUMENT CONFIGURATIONS
%----------------------------------------------------------------------------------------
\documentclass[11pt]{article}
\usepackage{amsmath} % Required for some math elements
\usepackage{hyperref} 
\usepackage{xcolor}
\usepackage{lipsum} 
\usepackage{cite}
\usepackage{graphicx} % Required for the inclusion of images
\usepackage{algorithmic}
\usepackage{array}
\usepackage{bookmark}
\usepackage{listings}
\usepackage{amssymb}
\usepackage{enumitem}
\usepackage[margin=24mm]{geometry}
\usepackage[caption=false, font=footnotesize]{subfig}
\usepackage{multirow}
\usepackage[active,tightpage]{preview}

\renewcommand{\PreviewBorder}{1in}
\newcommand{\Newpage}{\end{preview}\begin{preview}}

\newlist{steps}{enumerate}{1}
\setlist[steps, 1]{label = Step \arabic*:}

\hypersetup{ %color attributes of citation, link, etc.
    colorlinks=true,
    linkcolor=blue,
    filecolor=gray,      
    urlcolor=blue,
    citecolor=blue,
}

\newcommand{\matlab}{\textsc{Matlab }} %very important and totally necessary addition

\newcommand\Item[1][]{%
  \ifx\relax#1\relax  \item \else \item[#1] \fi
  \abovedisplayskip=0pt\abovedisplayshortskip=0pt~\vspace*{-\baselineskip}}
  %----------------------------------------------------------------------------------------
%	DOCUMENT INFORMATION
%----------------------------------------------------------------------------------------
 
\title{ECEN321 : Engineering Statistics \\ Assignment 8 Submission}
\author{Daniel Eisen : 300447549}
\date{\today}

\begin{document}
\begin{preview}
\maketitle
%----------------------------------------------------------------------------------------
%	DOCUMENT CONTENT
%----------------------------------------------------------------------------------------
\section*{Confidence Intervals}
\begin{enumerate}
        \item (Navidi 5.2.2), $n=100 \:\: x=73$ % 1
        \begin{enumerate}
                \item $\tilde{p} = \frac{x+2}{n+4} = 0.721153846154 \\
                z_{\alpha/2} = 1.96 \\
                \tilde{p} \pm z_{\alpha/2}\cdot\sqrt{\frac{\tilde{p}\left(1-\tilde{p}\right)}{n+4}} \\
                \tilde{p} \pm 0.086185795204 \Rightarrow (0.63496805095, 0.807339641358)$
        
                \item $\tilde{p} = \frac{x+2}{n+4} = 0.721153846154 \\
                z_{\alpha/2} = 2.575 \\
                \tilde{p} \pm z_{\alpha/2}\cdot\sqrt{\frac{\tilde{p}\left(1-\tilde{p}\right)}{n+4}} \\
                \tilde{p} \pm 0.113228787066 \Rightarrow (0.607925059087,0.83438263322)$

                \item $n=\left(\frac{1.96\sqrt{\tilde{p}\left(1-\tilde{p}\right)}}{0.05}\right)^{2}-4 \approx 305$\\ note as $\tilde{p}$ is relative to previous n, the actual required n is lower, but 305 will still bring the E below 0.05.
                
                \item $n=\left(\frac{2.575\sqrt{\tilde{p}\left(1-\tilde{p}\right)}}{0.05}\right)^{2}-4 \approx 530$\\ note as $\tilde{p}$ is relative to previous n, the actual required n is lower, but 530 will still bring the E below 0.05.
                
                \item  $p = 0.7, n=100, \tilde{p} = 0.72115\\
                z\ =\ \frac{p-\tilde{p}}{\sqrt{\frac{\tilde{p}\left(1-\tilde{p}\right)}{n+4}}} = -0.48 \\
                P(Z > -0.48) = .6844 = 68.44\%$
                
                \item $n = 200, p=0.95, k={193...200} \\
                P(X=k) = \frac{n!}{k!(n-k)!} \cdot p^k \cdot (1-p)^{n-k}\\
                P(X > 192) = \sum_{k=193}^{200}\frac{n!}{k!\cdot(n-k)!}\cdot p^{k}\cdot(1-p)^{n-k} = 0.2133$ \\
                I used wolfram to compute the sum, due to the large factorial.
        \end{enumerate}
        \item (Navidi  5.3.8) $\bar{X} = 3410.14,\: s = 1.018,\: n = 8,\: df\: = 7,\: CI=\bar{X}{\pm}t_{\alpha/2}{\cdot}\frac{s}{\sqrt{n}}$ % 2
        \begin{enumerate}
                \item $t_{\alpha/2} = 2.356 \\
                CI = 3410.14{\pm}2.356{\cdot}\frac{1.018}{\sqrt{8}} \Rightarrow (3409.29, 3410.99)$
                
                \item $t_{\alpha/2} = 2.998 \\
                CI = 3410.14{\pm}2.998{\cdot}\frac{1.018}{\sqrt{8}} \Rightarrow (3409.06, 3411.22)$
                
                \item No, because to able to use the student t CI's and calculations the samples must come from a population that is approximately normal, as seen by the outlier (3412.66) this sample cannot be said to come from a normal population so the above CI's cannot be used. 
        \end{enumerate}

        \item (Navidi  5.6.13) \\
                $$X = (207.4,\:233.1,\:215.9,\:235.1,\:225.6,\:244.4,\:245.3) $$
                $$\bar{X} = 229.54 ,\: s_X = 14.17 ,\: n_X = 7$$\\
                $$Y = (84.3,\:53.2,\:127.3,\:201.3,\:174.2,\:246.2,\:149.4,\:156.4,\:103.3) $$
                $$\bar{Y} = 143.96,\: s_Y = 59.76,\: n_Y = 9$$ \\
                $$CI = \bar{X}-\bar{Y}{\pm}z_{\alpha/2}{\cdot}\sqrt{\frac{s^2_X}{n_X} + \frac{s^2_Y}{n_Y}}, \:\:\: z_{\alpha/2} = 1.96$$
                $$85.58{\pm}1.96\cdot\sqrt{\frac{14.17^{2}}{7}+\frac{59.76^{2}}{9}} \approx 85.58{\pm}40.43$$
                $$CI = (45.15,\:126.01)$$
        
\end{enumerate}

\end{preview}
\end{document}