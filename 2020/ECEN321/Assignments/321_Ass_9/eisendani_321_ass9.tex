%----------------------------------------------------------------------------------------
%	PACKAGES AND DOCUMENT CONFIGURATIONS
%----------------------------------------------------------------------------------------
\documentclass[11pt]{article}
\usepackage{amsmath} % Required for some math elements
\usepackage{hyperref} 
\usepackage{xcolor}
\usepackage{lipsum} 
\usepackage{cite}
\usepackage{graphicx} % Required for the inclusion of images
\usepackage{algorithmic}
\usepackage{array}
\usepackage{bookmark}
\usepackage{listings}
\usepackage{amssymb}
\usepackage{enumitem}
\usepackage[margin=24mm]{geometry}
\usepackage[caption=false, font=footnotesize]{subfig}
\usepackage{multirow}
\usepackage[active,tightpage]{preview}

\renewcommand{\PreviewBorder}{1in}
\newcommand{\Newpage}{\end{preview}\begin{preview}}

\newlist{steps}{enumerate}{1}
\setlist[steps, 1]{label = Step \arabic*:}

\hypersetup{ %color attributes of citation, link, etc.
    colorlinks=true,
    linkcolor=blue,
    filecolor=gray,      
    urlcolor=blue,
    citecolor=blue,
}

\newcommand{\matlab}{\textsc{Matlab }} %very important and totally necessary addition

\newcommand\Item[1][]{%
  \ifx\relax#1\relax  \item \else \item[#1] \fi
  \abovedisplayskip=0pt\abovedisplayshortskip=0pt~\vspace*{-\baselineskip}}
  %----------------------------------------------------------------------------------------
%	DOCUMENT INFORMATION
%----------------------------------------------------------------------------------------
 
\title{ECEN321 : Engineering Statistics \\ Assignment 9 Submission}
\author{Daniel Eisen : 300447549}
\date{\today}

\begin{document}
\begin{preview}
\maketitle
%----------------------------------------------------------------------------------------
%	DOCUMENT CONTENT
%----------------------------------------------------------------------------------------
\section*{Hypothesis Tests}
\begin{enumerate}
        \item (Navidi 6.2.18) $98\%$ lower bound $= 50.1$, $H_0: \mu \le 50$, $H_1: \mu > 50$ \\
        That lower bounds tells us that there is a 2\% (0.02) chance of obtaining a sample mean more that 50.1.
        \begin{enumerate}
                \item Cannot determine if $P<0.01$ as we only know that$ P < 0.02$.
                \item As$ P < 0.02 < 0.05$ we can determine if $P < 0.05$.
        \end{enumerate}
        \item (Navidi 6.3.8) $n=300$, $x=12$ and $H_0: p \ge 0.08$, $H_1: p < 0.08$ \\

        $$z = \frac{\hat{p} - p_0}{\sqrt{p_0(1-p_0)/n}}$$
        $$\mathrm{from\;above: \;}p_0 = 0.08,\; \hat{p} = x/n = 12/300=0.04$$
        $$z = \frac{0.04-0.08}{\sqrt{0.08(1-0.08)/300}}=-2.55377$$
        $$P(Z<-2.55) = 0.0054$$

        This is (I think) sufficient evidence to reject the null hypothesis and support the claim of less then 8\% defective production.

        \item (Navidi 6.4.4) \\
        Ideal: 23 \\
        Sample: $n = 10$, $\bar{x}=23.2$, $s=0.2$
        \begin{enumerate}
                \item Null Hypothesis can be that the population mean \textbf{is} 23 and the alternate, that it is not. \\
                $H_0: \mu = 23$, $H_a: \mu \ne 23$ \\

                Test Statistic: $t = \frac{\bar{x}-\mu}{s/\sqrt{n}} = 0.2 / (0.2/\sqrt{10})=3.162278$
                $df = n-1 = 9$ \\

                \item From table with, a df of 9 and a t value of $\sim 3.16$ the two tailed P value is $\sim 0.0115$

                \item As this value represents the probability that the population mean is 23, I think there is sufficient evidence to claim that the process needs recalibration.
        \end{enumerate} 
\end{enumerate}

\end{preview}
\end{document}