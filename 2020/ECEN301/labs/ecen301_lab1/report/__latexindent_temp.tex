%----------------------------------------------------------------------------------------
%	PACKAGES AND DOCUMENT CONFIGURATIONS
%----------------------------------------------------------------------------------------
\documentclass[11pt]{article}
\usepackage{amsmath} % Required for some math elements
\usepackage{hyperref} 
\usepackage{xcolor}
\usepackage{lipsum} 
\usepackage{cite}
\usepackage{graphicx} % Required for the inclusion of images
\usepackage{algorithmic}
\usepackage{array}
\usepackage{bookmark}
\usepackage{listings}
\usepackage{amssymb}
\usepackage{enumitem}
\usepackage[margin=24mm]{geometry}
\usepackage[caption=false, font=footnotesize]{subfig}
\usepackage{multirow}
\usepackage[active,tightpage]{preview}

\renewcommand{\PreviewBorder}{1in}
\newcommand{\Newpage}{\end{preview}\begin{preview}}

\newlist{steps}{enumerate}{1}
\setlist[steps, 1]{label = Step \arabic*:}

\hypersetup{ %color attributes of citation, link, etc.
    colorlinks=true,
    linkcolor=blue,
    filecolor=gray,      
    urlcolor=blue,
    citecolor=blue,
}

\newcommand{\matlab}{\textsc{Matlab }} %very important and totally necessary addition

\newcommand\Item[1][]{%
  \ifx\relax#1\relax  \item \else \item[#1] \fi
  \abovedisplayskip=0pt\abovedisplayshortskip=0pt~\vspace*{-\baselineskip}}
  %----------------------------------------------------------------------------------------
%	DOCUMENT INFORMATION
%----------------------------------------------------------------------------------------
 
\title{ECEN301 : Embedded Systems \\ Lab 1 Submission}
\author{Daniel Eisen : 300447549}
\date{\today}

\begin{document}
\begin{preview}
\maketitle
%----------------------------------------------------------------------------------------
%	DOCUMENT CONTENT
%----------------------------------------------------------------------------------------
\section{Objectives}
The lab was to learn how to use Atmel Studio to develop c code for microcontroller (specifically the 8051), developing an understanding of how to control sending data in and out to the pins/ports and using a series of peripheral inputs. This is meant to be the basis for future labs.

\section{Methodology}
        \subsection{Introduction}
        To begin with, setting up a connection to, and confirming a c programming connection. Initially there was a problem with this, flashing that is, was Atmel was unable to automatically use BATCHISP to program after build. The solution was to "manually" connect and go through the programming steps using FLIP; establishing a COM port connection, loading hex file, and flashing the 8051. 

        \subsubsection{Using ports to output data}
        The first task was basic input and output to and from the ports of the $\mu$C. This was done initially by loading a count.c program to display to the LED in the IO module. I was then observed that with no delay it was impossible to read without, for example slowing a recording down etc, so a delay function was used. Basically a non useful for loop.
        
        This did however have a limit do to overflowing the passed 16 bit integer above 32768, so multiple calls are required for more delay. This could be improved by unsung a timer based delay.
        
        Next was to toggle the output between 0b00001111 and 0b11110000. Again using a delay for human readability. This output was connected to the IO module to display on the LED.

        Finally, to establish human input and $\mu$C output, two IO modules were used. One in EXT mode to receive input from P2 and one in INT mode to send data to P1. The software then connected the dataflow between these 2 ports so the input switches of one module controlled the LED output of the other.


        \subsubsection{Keyboard scanning}

        \begin{lstlisting}[language=C]
        #include "AT89C51AC3.h"
        #include "ECEN301LibSDCC.h"
        
        void main()
        {
                unsigned char key;
                while (1)
                {
                        key = sampleKeypad();
                        P1 = key;
                }
        }
        \end{lstlisting}

        \subsubsection{Digital to Analogue Conversion (DAC)}

        \begin{lstlisting}
                #include "AT89C51AC3.h"
                #include <math.h>
                
                #define STEP 2048.0
                
                void main(void)
                {
                        /* Replace with your application code */
                        while (1)
                        {
                                for (int i = 0; i < STEP; ++i){
                                        int val = 127.0 * sinf(((2*PI)/STEP)*i) + 128.0;
                                        P1 = val;
                                }
                        }
                }
                        
        \end{lstlisting}

        \subsubsection{DAC Motor Control}
\section{Questions}
\begin{enumerate}
        \item The Object and executable files and object files are generated on compilation. The executable binary is the machine code that will be read and executed the controllers CPU and not really readable or understandable to us. The object file is not executable but is used only in compilation during the linking stage. The rel file is a generated object file from the assembler per src file, this is feed into the linage editor.
        \item Port 0-3 can be fully used, with all 8-bits accessible. But P4.0 and P4.1 are the data lines of the UART connection the computer for programming, the switch (dis)connects these pins from the USB interface.
        \item 
        \item 
        \item 
        \item 
        \begin{enumerate}
                \item 
                \item 
                \item 
        \end{enumerate}
\end{enumerate}

\end{preview}
\end{document}