%----------------------------------------------------------------------------------------
%	PACKAGES AND DOCUMENT CONFIGURATIONS
%----------------------------------------------------------------------------------------
\documentclass[11pt]{article}
\usepackage{amsmath} % Required for some math elements
\usepackage{hyperref} 
\usepackage{xcolor}
\usepackage{lipsum} 
\usepackage{cite}
\usepackage{graphicx} % Required for the inclusion of images
\usepackage{algorithmic}
\usepackage{array}
\usepackage{bookmark}
\usepackage{listings}
\usepackage{amssymb}
\usepackage{enumitem}
\usepackage[margin=24mm]{geometry}
\usepackage[caption=false, font=footnotesize]{subfig}
\usepackage{multirow}
\usepackage[active,tightpage]{preview}

\renewcommand{\PreviewBorder}{1in}
\newcommand{\Newpage}{\end{preview}\begin{preview}}

\newlist{steps}{enumerate}{1}
\setlist[steps, 1]{label = Step \arabic*:}

\hypersetup{ %color attributes of citation, link, etc.
    colorlinks=true,
    linkcolor=blue,
    filecolor=gray,      
    urlcolor=blue,
    citecolor=blue,
}


\definecolor{mGreen}{rgb}{0,0.7,0.5}
\definecolor{mWhite}{rgb}{0.9,0.9,0.9}
\definecolor{mGray}{rgb}{0.5,0.5,0.5}
\definecolor{mPurple}{rgb}{0.58,0,0.82}
\definecolor{backgroundColour}{rgb}{0.0,0.0,0.1}

\lstdefinestyle{Cstyle}{
    backgroundcolor=\color{backgroundColour},   
    commentstyle=\color{mGreen},
    keywordstyle=\color{magenta},
    numberstyle=\tiny\color{mGray},
    stringstyle=\color{mPurple},
    basicstyle=\footnotesize\color{mWhite},
    breakatwhitespace=false,         
    breaklines=true,                 
    captionpos=b,                    
    keepspaces=true,                 
    numbers=left,                    
    numbersep=5pt,                  
    showspaces=false,                
    showstringspaces=false,
    showtabs=false,                  
    tabsize=4,
    language=C
}


\newcommand{\matlab}{\textsc{Matlab }} %very important and totally necessary addition

\newcommand\Item[1][]{%
  \ifx\relax#1\relax  \item \else \item[#1] \fi
  \abovedisplayskip=0pt\abovedisplayshortskip=0pt~\vspace*{-\baselineskip}}
  %----------------------------------------------------------------------------------------
%	DOCUMENT INFORMATION
%----------------------------------------------------------------------------------------
 
\title{ECEN301 Embedded Systems Lab 6 \\ JTAG Debugging}
\author{Daniel Eisen 300447549}
\date{\today}

\begin{document}
\begin{preview}
\maketitle
%----------------------------------------------------------------------------------------
%	DOCUMENT CONTENT
%----------------------------------------------------------------------------------------
\section{Objectives}
So far we have only used the JTAG port for upload our code to the ARM device. But the port/interface and associated statemachine and capabilities have a far greater use. In this lab we investigated the direct register read/write feature, the use of step by step and breakpoint debugging and observing the affect of the ARM(+thumb) instruction sets optional suffix's on operational register flags. 
\section{Methodology}
To begin with 
\section{Questions}
\begin{enumerate}
    \item \textbf{\textit{What do you think this code does?}}\\
    These lines of code are part of the setup for the volatile variables.\\
    \texttt{str lr, [sp, \#-4]!} is a STORE with immediate offset (pre-indexed), storing the contents of the \textbf{lr} register in 4 places before the stack pointer.

    \texttt{80000498: E24DD00C sub sp, sp, \#0xc} This subtracts 0xC from the stored in the stack pointer and stores the result in back in sp.
    \item \textbf{\textit{What is the difference in behaviour between the ADDS and ADD instructions?}}\\
    Both instructions are the basic Add without carry operation. The difference being the optional suffix 'S', with this present the ADDS instruction (and any instruction with the S suffix) updates the N,Z,X,V flags in the Current Program Status Register (CPSR).
    \item \textbf{\textit{List the debug capabilities of the JTAG port.}}\\
    The use of the JTAG port in debugging are: 
    
    Traditional external connection testing of the chip. This can serve as a lower barrier to entry method of formfilling a similar role of a fully blown logic analyser in debugging the communions/connection between the device and its peripherals.

    Additionally, due to the JTAG's interface being able to directly read/write a devices registers it can also be used as a means of programming the flash, and enabling step by step debugging and breakpoints via the TAP controller/FSM.
\end{enumerate}
\end{preview}
\end{document}