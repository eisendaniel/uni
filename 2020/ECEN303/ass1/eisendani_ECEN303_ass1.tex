%----------------------------------------------------------------------------------------
%	PACKAGES AND DOCUMENT CONFIGURATIONS
%----------------------------------------------------------------------------------------
\documentclass[11pt]{article}
\usepackage{amsmath} % Required for some math elements
\usepackage{hyperref} 
\usepackage{xcolor}
\usepackage{lipsum} 
\usepackage{cite}
\usepackage{graphicx} % Required for the inclusion of images
\usepackage{algorithmic}
\usepackage{array}
\usepackage{bookmark}
\usepackage{listings}
\usepackage{amssymb}
\usepackage{enumitem}   
\usepackage[margin=16mm]{geometry}
\usepackage[caption=false, font=footnotesize]{subfig}

\newlist{steps}{enumerate}{1}
\setlist[steps, 1]{label = Step \arabic*:}

\hypersetup{ %color attributes of citation, link, etc.
    colorlinks=true,
    linkcolor=blue,
    filecolor=gray,      
    urlcolor=blue,
    citecolor=blue,
}

\newcommand{\matlab}{\textsc{Matlab}} %very important and totally necessary addition

\newcommand\Item[1][]{%
  \ifx\relax#1\relax  \item \else \item[#1] \fi
  \abovedisplayskip=0pt\abovedisplayshortskip=0pt~\vspace*{-\baselineskip}}
%----------------------------------------------------------------------------------------
%	DOCUMENT INFORMATION
%----------------------------------------------------------------------------------------
 
\title{ECEN321: Analogue Electronics \\ Assignment 1: Power Amplifiers - Submission}
\author{Daniel Eisen : 300447549}
\date{\today}

\begin{document}
\maketitle
%----------------------------------------------------------------------------------------
%	DOCUMENT CONTENT
%----------------------------------------------------------------------------------------
\section*{Question 1}
  \begin{enumerate}[label = \Roman*.]
          \item \textit{An audio amplifier operates in the frequency range of..} \\ 
          \textbf{a. 20Hz to 20kHz} 

          \item \textit{For maximum peak-to-peak output voltage, the Q point should be..} \\ 
          \textbf{c. At the centre of the dc load line}
          
          \item \textit{An amplifier has two load lines because..} \\ 
          \textbf{d. All of the above} 
          
          \item \textit{Push-pull is almost always used with..} \\ 
          \textbf{b. Class B} 
          
          \item \textit{Class C amplifiers are almost always..} \\ 
          \textbf{c. Tuned RF amplifiers} 
          
          \item \textit{The input signal of a class C amplifier..} \\ 
          \textbf{c. Produces brief pulses of collector current} 
          
          \item \textit{If RC=100 Ωand RL=180Ω, the ac load resistance equals..} \\ 
          \textbf{The answer} 
          
          \item \textit{In a class A amplifier, the collector current flows for..} \\ 
          \textbf{d. The entire cycle} 
          
          \item \textit{With class A, the output signal should be..} \\ 
          \textbf{a. Unclipped} 
          
          \item \textit{A small quiescent current is necessary with a class AB push-pull amplifier to avoid..} \\ 
          \textbf{a. Crossover distortion} 
  \end{enumerate}

\section*{Question 2}
  \begin{enumerate}[label=\alph*)]
    \item % a)
    In the push push-pull configuration, the base-emitter junctions of the transistors have a potential of 0.7V. Thus when the input drops below this, at the \textbf{crossover} around the zero-point, the output is cut-off. This introduces a flatline distortion between positive and negative half cycles.

    \item % b)
    

  \end{enumerate}

\section*{Question 3}

\section*{Question 4}
  \begin{enumerate}[label=\roman*)]
    \item % maximum  a.c. power output
    \item % the power rating of transistor
    \item % he maximum efficiency.
  \end{enumerate}

\section*{Question 5}

\section*{Question 6}

\end{document}