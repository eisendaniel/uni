%----------------------------------------------------------------------------------------
%	PACKAGES AND DOCUMENT CONFIGURATIONS
%----------------------------------------------------------------------------------------
\documentclass[11pt]{article}
\usepackage{amsmath} % Required for some math elements
\usepackage[usenames,dvipsnames]{xcolor}
\usepackage{lipsum} 
\usepackage{cite}
\usepackage{graphicx} % Required for the inclusion of images
\usepackage{algorithmic}
\usepackage{array}
\usepackage{bookmark}
\usepackage{listings}
\usepackage{amssymb}
\usepackage{enumitem}
\usepackage[margin=24mm]{geometry}
\usepackage[caption=false, font=footnotesize]{subfig}
\usepackage{multirow}
\usepackage[active,tightpage]{preview}
\usepackage{hyperref} 

\renewcommand{\PreviewBorder}{1in}
\newcommand{\Newpage}{\end{preview}\begin{preview}}

\newlist{steps}{enumerate}{1}
\setlist[steps, 1]{label = Step \arabic*:}

\hypersetup{ %color attributes of citation, link, etc.
    colorlinks=true,
    linkcolor=blue,
    filecolor=gray,      
    urlcolor=blue,
    citecolor=blue,
}

 
\lstdefinelanguage{VHDL}{
    morekeywords=[1]{
        library,use,all,entity,is,port,in,out,end,architecture,of,begin,and,or,Not,downto,ALL
    },
    morekeywords=[2]{
        STD_LOGIC_VECTOR,STD_LOGIC,IEEE,STD_LOGIC_1164,NUMERIC_STD,STD_LOGIC_ARITH,STD_LOGIC_UNSIGNED,std_logic_vector,std_logic
    },
    morecomment=[l]--
}

\definecolor{keyword}{rgb}{0,0.3,0.7}
\definecolor{STD}{rgb}{0.9,0.0,0.7}
\definecolor{comment}{rgb}{0.0,0.6,0.1}

\lstdefinestyle{vhdl}{
   language     = VHDL,
   basicstyle   = \footnotesize\ttfamily,
   keywordstyle = [1]\color{keyword}\bfseries,
   keywordstyle = [2]\color{STD}\bfseries,
   commentstyle = \color{comment}
   breaklines=true,                % sets automatic line breaking
   tabsize=3		                   % sets default tabsize to 2 spaces
}


\newcommand{\matlab}{\textsc{Matlab }} %very important and totally necessary addition

\newcommand\Item[1][]{%
    \ifx\relax#1\relax  \item \else \item[#1] \fi
    \abovedisplayskip=0pt\abovedisplayshortskip=0pt~\vspace*{-\baselineskip}}
%----------------------------------------------------------------------------------------
%	DOCUMENT INFORMATION
%----------------------------------------------------------------------------------------
 
\title{ECEN302 : Embedded Systems \\ Assignment 2 Submission}
\author{Daniel Eisen : 300447549}
\date{\today}

\begin{document}
\begin{preview}
\maketitle
%----------------------------------------------------------------------------------------
%	DOCUMENT CONTENT
%----------------------------------------------------------------------------------------
\begin{enumerate}
    \item \textit{\textbf{Explain how a Cache works and how it improves the performance of a microprocessor.}}
    
    \item \textit{\textbf{What is the main purpose of a Memory Management Unit?}}
    
    \item \textit{\textbf{What does the acronym JTAG stand for? Explain the uses of a JTAG interface.}}
    
    \item \textit{\textbf{Explain at least 4 of the many things that need to be considered when developing an embedded system for a product.}}
    
    \item \textit{\textbf{Explain 4 things you can do with a JTAG unit while debugging code.}}
    
    \item \textit{\textbf{ADCs and other devices often communicate with the microprocessor via a serial interface. What do the acronyms I2C and SPI stand for? Explain the differences between these two interface methods.}}
    
    \item \textit{\textbf{The first time an optical shaft encoder was connected to a microprocessor the engineer used an external interrupt but with it configured as a level edge triggered interrupt. Explain why that was not a good idea.}}
    
    \item \textit{\textbf{What does the acronym PCIe stand for? Give an overview of how this interface works and how it uses the advantages of both parallel and serial communications.}}
    
    \item \textit{\textbf{Discuss the differences between “bare metal” and a Linux operating system and give an example of when each is more appropriate.}}


\end{enumerate}

\end{preview}
\end{document}  