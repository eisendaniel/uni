%----------------------------------------------------------------------------------------
%	PACKAGES AND DOCUMENT CONFIGURATIONS
%----------------------------------------------------------------------------------------
\documentclass[11pt]{article}
\usepackage{amsmath} % Required for some math elements
\usepackage[usenames,dvipsnames]{xcolor}
\usepackage{lipsum} 
\usepackage{cite}
\usepackage{graphicx} % Required for the inclusion of images
\usepackage{algorithmic}
\usepackage{array}
\usepackage{bookmark}
\usepackage{listings}
\usepackage{amssymb}
\usepackage{enumitem}
\usepackage[margin=24mm]{geometry}
\usepackage[caption=false, font=footnotesize]{subfig}
\usepackage{multirow}
\usepackage[active,tightpage]{preview}
\usepackage{hyperref} 

\renewcommand{\PreviewBorder}{1in}
\newcommand{\Newpage}{\end{preview}\begin{preview}}

\newlist{steps}{enumerate}{1}
\setlist[steps, 1]{label = Step \arabic*:}

\hypersetup{ %color attributes of citation, link, etc.
    colorlinks=true,
    linkcolor=blue,
    filecolor=gray,      
    urlcolor=blue,
    citecolor=blue,
}

 
\lstdefinelanguage{VHDL}{
    morekeywords=[1]{
        library,use,all,entity,is,port,in,out,end,architecture,of,begin,and,or,Not,downto,ALL
    },
    morekeywords=[2]{
        STD_LOGIC_VECTOR,STD_LOGIC,IEEE,STD_LOGIC_1164,NUMERIC_STD,STD_LOGIC_ARITH,STD_LOGIC_UNSIGNED,std_logic_vector,std_logic
    },
    morecomment=[l]--
}

\definecolor{keyword}{rgb}{0,0.3,0.7}
\definecolor{STD}{rgb}{0.9,0.0,0.7}
\definecolor{comment}{rgb}{0.0,0.6,0.1}

\lstdefinestyle{vhdl}{
   language     = VHDL,
   basicstyle   = \footnotesize\ttfamily,
   keywordstyle = [1]\color{keyword}\bfseries,
   keywordstyle = [2]\color{STD}\bfseries,
   commentstyle = \color{comment}
   breaklines=true,                % sets automatic line breaking
   tabsize=3		                   % sets default tabsize to 2 spaces
}


\newcommand{\matlab}{\textsc{Matlab }} %very important and totally necessary addition

\newcommand\Item[1][]{%
    \ifx\relax#1\relax  \item \else \item[#1] \fi
    \abovedisplayskip=0pt\abovedisplayshortskip=0pt~\vspace*{-\baselineskip}}
%----------------------------------------------------------------------------------------
%	DOCUMENT INFORMATION
%----------------------------------------------------------------------------------------
 
\title{ECEN302 : Integrated Digital Electronics \\ Assignment 2 Submission}
\author{Daniel Eisen : 300447549}
\date{\today}

\begin{document}
\begin{preview}
\maketitle
%----------------------------------------------------------------------------------------
%	DOCUMENT CONTENT
%----------------------------------------------------------------------------------------
\begin{enumerate}
    \item \textit{\textbf{List three advantages of scaling down the feature sizes of silicon devices.}}
    
    \begin{itemize}
        \item Higher density means more transistors on a single device
        \item Smaller distance means faster propagation tome and lower power loss.
        \item Smaller sized dies can be run faster and cooler 
    \end{itemize}
    
    \item \textit{\textbf{List two consequences of scaling down the feature sizes of silicon devices.}}
    
    \begin{itemize}
        \item As feature sizes decrease the sum effect of slow atomic diffusion through the semiconductor material decrease the time until the device is unusable.
        \item At smaller and smaller "trace" sizes the risk/probability of electrons quantum tunnelling becomes significant.  
    \end{itemize}
    
    \item \textit{\textbf{Briefly discuss and compare the performance and typical uses of microprocessors and FPGAs.}}
    
    
    
    \item \textit{\textbf{List four advantages of integrating a microprocessor and an FPGA onto a single chip.}}
    
    \begin{itemize}
        \item 
        \item 
        \item 
    \end{itemize}
    
    \item \textit{\textbf{Provide one application or product example that benefits from having both a microprocessor and a FPGA.}}

    \item \textit{\textbf{In a RF receiver signal chain, why is it advantageous to have the ADC as close as possible to the Antenna?}}
    
    \item \textit{\textbf{Describe the operation of the OSERDES and ISERDES FPGA I/O blocks.}}
    
    \item \textit{\textbf{Describe, with the aid of diagrams, how you would connect the AD9739 DAC to a Xilinx 7 series device and run the DAC at 2GSPS (note: you do not need to create a detailed schematic diagram).}}
    

\end{enumerate}

\end{preview}
\end{document}  