%----------------------------------------------------------------------------------------
%	PACKAGES AND DOCUMENT CONFIGURATIONS
%----------------------------------------------------------------------------------------

\documentclass[12pt,a4paper]{article}

\usepackage{amsmath} % Required for some math elements 
\usepackage[margin=32mm]{geometry}


\addtolength{\topmargin}{-16mm}
\addtolength{\textheight}{32mm}


\setlength\parindent{1pt} % Removes all indentation from paragraphs
\renewcommand{\labelenumi}{\alph{enumi}.} % Make numbering in the enumerate environment by letter rather than number (e.g. section 6)

\usepackage{setspace}
\renewcommand{\baselinestretch}{1.5} 
%----------------------------------------------------------------------------------------
%	DOCUMENT INFORMATION
%----------------------------------------------------------------------------------------

\title{201 HW 3}
\author{Daniel Eisen : 300447549}
\date{\today}

\begin{document}
\maketitle

\section{Reducing Human Radiation Risks}
Human undertaking space travel and extraterrestrial habitation outside of a protective magnetosphere are subject to very high levels of highly damaging radiation [1]. Exposure levels are at least 3 times that experienced in low earth orbit (LEO) [2]. This radiation is of two distinct forms, Galactic Cosmic Rays (GCR)and Solar Particle Events (SPE). With equally distinct needed approach to shielding/risk reduction.
 
\subsection{Requirements for risk mitigation }
The radiation risk evaluation and mitigation is if the highest importance, as current "Mars mission models conclude that astronauts will vastly exceed the current limit for a career-based 3 percent increase in Risk of Exposure Induced Death (REID) over the course of a typical mission.[3]" Especially to vital organs, i.e. lungs and with a higher risk to female reproductive organs[3]. To have a mission be sustainable exposure must be brought to what is refereed to by OSHA and NASA as, "as low as reasonably achievable", or
ALARA[4].

\subsection{Shielding technologies to mitigate GCR/SPE Exposure}
Three shield main shielding technologies have been investigated in [5]; aluminium, water, and liquid hydrogen all reduce GCR damage to BFO (blood forming organs) to 45 $rem.yr^{-1}$, $35 rem.yr^{-1}$ and $15 rem.yr^{-1}$ respectively at a minimum of 10 $g.cm^{-3}$ [5] \\ Where SPE are unpredictable both in time and power so a higher grade of aluminium shield, nearer 30 $g.cm^{-3}$ is needed to have same risk reduction.. Though the practicality is also a major factor. Mostly the concerns of each of the weight requirement of each, as the liquids and aluminium as sufficiently effective thickness's are impractically heavy when used on their own. As well as requiring added structural support [1].
This is where nano tube technologies are particularly promising in part because of
their potential cross-application as structural materials as well. 
Tables 2-4 in [1] investigate their respective effectivenesses and show promising results:

\begin{center}
Nanoporous carbon composites (CNTs).\\ 
Specific “best case” example under comparison:
(C2H4)39.13 \%(CH3)60.87% 

Hydrogen-loaded metal organic frameworks (MOFs).\\
Specific “best case” example under comparison:
$C_{432}H_{1120}Be_{48}O_{144}$

Hydrogenated boron nitride nanotubes (BNNTs).\\
Specific “best case” example under comparison: BNNT
+ 20\% by weight H2 
\end{center}
\subsection{Colony Design}
In an option to reduce cost and extensive use of external shielding, the Martian regolith, ie in situ (underground) shielding can be used. 

In terms of architecture, the base must be constructed to withstand a worst case SPE for long stay missions (~600 days). This can be mitigated with base location, to reduce direct radiation exposure. Making use of geography and solar angle to reduce radiation intensity. In terms of radiation-effect mitigation, the mission must be supply with adequate medical supplies to cover the expected length and worst case exposure, (within planed protocols).
 
To have a successful human mission to Mars, with the objective to maintain crew health and safety, the mission must; maintain radiation exposure to acceptable levels, and mitigate effects of suffered exposure.

\begin{center}
%----------------------------------------------------------------------------------------
%	REFERENCES
%----------------------------------------------------------------------------------------

\newpage
\section*{References}
\end{center}
{[1]} K. A. Worden-Buckner, S. Tackett, J. L. Rhatigan, M. Rhoades, "Reducing Human Radiation Risks on Deep Space Missions", 2018 IEEE Aerospace Conference \\

{[2]} F. A. Cucinotta, M. H. Y. Kim, L. J. Chappell and J. L.
Huff, "How Safe Is Safe Enough? Radiation Risk for a
Human Mission to Mars," PLoS ONE, vol. 8, no. 10,
pp. 1-9, 2013 \\

{[3]} F. A. Cucinotta and M. Durant, "Risk of Radiation
Carcinogenesis," in Human Health and Performance
Risks of Space Exploration Missions, J. C. M. a. J. B.
Charles, Ed., Washington, DC: NASA, 2010, pp. 119-
169. \\

{[4]} A. M. Sieffert, "Astronaut Health \& Safety Regulations
Ionizing Radiation," The SciTech Lawyer, vol. 10, no.
4, pp. 20-22, 2014 \\

{[5]} K. Worden, "Reducing Human Radiation Risks on
Deep Space Missions," Naval Postgraduate School,
Monterey, CA, 2017 \\
\end{document}