%----------------------------------------------------------------------------------------
%	PACKAGES AND DOCUMENT CONFIGURATIONS
%----------------------------------------------------------------------------------------

\documentclass[11pt]{article}

\usepackage{amsmath} % Required for some math elements 
\usepackage[margin=32mm]{geometry}


\addtolength{\topmargin}{-16mm}
\addtolength{\textheight}{32mm}


\setlength\parindent{1pt} % Removes all indentation from paragraphs
\renewcommand{\labelenumi}{\alph{enumi}.} % Make numbering in the enumerate environment by letter rather than number (e.g. section 6)


%----------------------------------------------------------------------------------------
%	DOCUMENT INFORMATION
%----------------------------------------------------------------------------------------

\title{201 HW 2}
\author{Daniel Eisen : 300447549}
\date{\today}

\begin{document}
\maketitle
\begin{center}
\section*{Summary}
\end{center}

 This paper {[1]} outlines a systematic analysis of the risks of radiation exposure to humans, specifically from the the point of reference of a long term space mission. More specifically what is the exposure, how it can mitigated and finally what would be optimal for a the design of Martian colony.
 
 Astronauts on deep space voyages and by extension on a Martian expedition would be subjected to very high levels of both Galactic Cosmic Rays (GCR)and Solar Particle Events (SPE) and while shielding exists (ie the ISS) exposure level exceed those design specifications be at least 3 times.
 
 The radiation risk evaluation and mitigation is if the highest importance, as current "Mars mission models conclude that astronauts will vastly exceed the current limit for a career-based 3 percent increase in Risk of Exposure Induced Death (REID) over the course of a typical mission."  3 shield main shielding technologies, aluminium, water, and liquid hydrogen all reduce GCR damage to BFO (blood forming organs) to 45 rem/yr, 35 rem/yr and 15 rem/yr respectively at a minimum of 10 $g/cm^{3}$. Where SPE are unpredictable both in time and power so a higher grade of aluminium shield, nearer 30 $g/cm^{3}$ is needed to have same risk reduction.
 
 In terms of architecture, the base must be constructed to withstand a worst case SPE for long stay missions (~600 days). This can be mitigated with base location, to reduce direct radiation exposure. Making use of geography and solar angle to reduce radiation intensity. In terms of radiation-effect mitigation, the mission must be supply with adequate medical supplies to cover the expected length and worst case exposure, (within planed protocols).
 
To have a successful human mission to Mars, with the objective to maintain crew health and safety, the mission must; maintain radiation exposure to acceptable levels, and mitigate effects of suffered exposure.\\ 


\begin{center}
\section*{Evaluation} 
\end{center}
This paper has extensive and detailed coverage and in depth analysis of risk and good solution comparison. However the extent of the mission length it covers if 600 days, without guarantee that conclusions can be extrapolated to missions of greater length.

\newpage
\begin{center}
\section*{References}
\end{center}
{[1]} Kathryn A. Worden-Buckner, Stephen Tackett, Jennifer L. Rhatigan, Mark Rhoades, "Reducing Human Radiation Risks on Deep Space Missions", 2018 IEEE Aerospace Conference

\end{document}