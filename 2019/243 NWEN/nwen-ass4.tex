%----------------------------------------------------------------------------------------
%	PACKAGES AND DOCUMENT CONFIGURATIONS
%----------------------------------------------------------------------------------------

\documentclass[a4paper,11pt]{article}

\usepackage{hyperref}
\usepackage{xcolor}
\usepackage{graphicx}
\usepackage{amsmath}
\usepackage[margin=16mm]{geometry}

\setlength\parindent{0pt} % Removes all indentation from paragraphs
%\renewcommand{\labelenumi}{\alph{enumi}.} % Make numbering in the enumerate environment by letter rather than number (e.g. section 6)

%---------------------------------------------------------------------------------------
%	DOCUMENT INFORMATION
%---------------------------------------------------------------------------------------

\title{NWEN 243 : Assignment 4}
\author{Daniel Eisen : 300447549}
\date{\today}

\begin{document}
\maketitle

%---------------------------------------------------------------------------------------
% DOCUMENT CONTENTS
%---------------------------------------------------------------------------------------
\begin{enumerate}
%1
\item
\textcolor{gray}{\textit{Briefly describe what DHCP and NAT are used for.}}\\ \\
DHCP, the Dynamic Host Configuration Protocol, is uses on a network to dynamically assign IP address (plus other network configs) to a device on the network.
\\
NAT is a method of mapping one IP address (of a local networks router) to a range of addresses within the network.
%2
\item
\textcolor{gray}{\textit{Briefly describe how Dijkstra’s algorithm works.}}\\ \\
It's an algorithm to find the shortest path between a and b. It picks the unvisited node with the low distance, calculates the distance through it to each unvisited neighbor, and updates the neighbor's distance if smaller.
%3
\item
\textcolor{gray}{\textit{Briefly describe how Distance Vector algorithm works.}}
\begin{tabbing}
A \= router transmits its distance vector to each of its neighbors.\\
Each router saves the most recently received distance vector from each of its neighbors.\\
A router recalculates its distance vector when:\\
\> It receives a distance vector from a neighbor containing new info.\\
\> It discovers that a link to a neighbor has gone down.\\
The DV equation is based on minimizing the cost to each destination
\end{tabbing}
%4
\item
\textcolor{gray}{\textit{Name two examples of IGP.}}
\begin{itemize}
	\item RIP: Routing Information Protocol
	\item OSPF: Open Shortest Path First
\end{itemize}
%5
\item
\textcolor{gray}{\textit{In relation to AS, briefly compare what IGP and BGP are
used for.}}\\ \\
IGP is the protocol use to route traffic within an autonomous systems.\\
Where is BGP is the protocol for routing traffic between different autonomous systems.
%6
\item
\textcolor{gray}{\textit{In BGP, explain what eBGP and iBGP do.}}
\begin{itemize}
	\item eBGP: obtain subnet reachability information from neighboring ASs.
	\item iBGP: propagate reachability information to all AS-internal routers.
\end{itemize}
%7
\item
\textcolor{gray}{\textit{In BGP, explain how the shortest path is determined and
how to identify the border gateway router (the router connecting
to the next-hop AS).}}\\ \\
The shortest path is found with OSPF, utilizing Dijkstra's and the Distance Vector algorithm.
The border gateway router of an AS is that/those through which packets leave the AS to other AS's etc..
%8
\item
\textcolor{gray}{\textit{In BGP, if there is a tie between two paths with the
same cost / distance, explain how the path will be selected.}}\\ \\
It will choose the path with the closest NEXT-HOP.
%9
\item
\textcolor{gray}{\textit{Each network interface card has a permanent address.
What is this address called? In terms of routing what is this
address used for?}}\\ \\
This permanent address is the MAC address. This is used in frame headers to identify source and destination.
%10
\item
\textcolor{gray}{\textit{Based on parity check, explain how a single bit
error can be detected and corrected.}}\\ \\
Based on either a checksum or 1-2 stored parity bits the value of an aligned data bit can be determined to be integral of not, ie corrupted.
%11
\item
\textcolor{gray}{\textit{In addition to parity check, why was CRC proposed?
Briefly explain how CRC works.}}\\ \\
CRC is a more powerful and robust error-detection coding:
\\
Calculates a short, binary sequence, a CRC, for each block of data to be sent or stored and appends it to the data, forming a codeword.

When a codeword is received or read, the device either compares its check value with one freshly calculated from the data block, or equivalently, performs a CRC on the whole codeword and compares the resulting check value with an expected residue constant.

If the CRC values do not match, then the block contains a data error
%12
\item
\textcolor{gray}{\textit{Name two of the fundamental MAC protocols covered in
the lecture, which use channel partitioning or random access, and
briefly explain how they work.}}\\ \\
TDMA, time division multiple access is a channel partitioning MAC protocol. \\
Channel is accessed in rounds, and each station gets fixed time length per round.
FDMA, frequency division multiple access is also a channel partitioning protocol.\\
Channel is divided into frequency bands, where each station has a set bandwidth allocation for access.
%13
\item
\textcolor{gray}{\textit{Explain how ARP works and how the forwarding table
in a switch (not a router) is created.}}\\ \\
Address Resolution Protocol is a procedure for mapping a IP address to a MAC address.

\end{enumerate}
\end{document}
